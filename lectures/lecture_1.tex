\section{Lecture I}
\subsection{Simple Rules and Complex Outcomes}
\textbf{Software is created for a purpose (aka requirements)}
\begin{itemize}
    \item Complexity due to the interactions between simple steps, which \textbf{became unpredictable} as the \textbf{software scales}
    \item \textbf{Human Limitations}: Programs that can be fully \underline{comprehended} can be created, especially in \underline{large}, \underline{changing} teams.
\end{itemize}

\subsection{The Challenge of Evolving Software Requirements}
\begin{itemize}
    \item Software development is a \textbf{dynamic process} with evolving requirement

    \item The evolution necessitates \underline{flexible}, \underline{adaptable} designs that can be able to vary without major overhauls
\end{itemize}

\begin{example}
    An e-commerce app focused on \textit{product listing} may need to \textbf{integrate} advanced features like \underline{AI-based recommendations} or \underline{AR previews} as market and business demands evolve.
\end{example}

\subsection{Designing for Complexity and Change}
\begin{itemize}
    \item Design principles are essential to manage software \textbf{complexity} and \textbf{changing requirements}.
    \item Object-oriented design offers techniques like information hiding, interfaces and polymorphism to promote \gls{loose_coupling}.
\end{itemize}

\subsection{Good Software - Beyond Correctness}
\begin{itemize}
    \item Good Software is not just about being correct!
    \item Designing software involves making choices concerning \gls{efficiency_and_maintainability}.
\end{itemize}

\subsection{Design as an Art and Science}
\begin{itemize}
    \item Software design is a much an art as it is a science
\end{itemize}

\subsection{How to design good software}

\begin{example} 
    \textbf{Ride Sharing Apps}: Uber, Bolt and Lyft Apps need to adapt to new technologies and user expectations to be competitive. \\
    \textbf{Uber's Platform} \\
    \begin{itemize}
        \item \textbf{Initial Approach:} Began as a service for booking luxury car rides in San Francisco.
        \item \textbf{Evolution:} Expanded operations to over 900 metropolitan areas offering other services.
        \item \textbf{Diversification:} Included additional services like UberX, UberPool and UberEats to meet diverse consumer needs.
    \end{itemize}    
\end{example}


\subsection{So how do Uber and other companies design good software}
With the principles and practices of Software Engineering, a discipline dedicated to crafting high-quality software!


\subsection{Software Engineering}
Software Engineering is the application of engineering principles to software development in a systematic way. 
The goal is to produce reliable, efficient, maintainable and usable software.

\subsubsection{The Essence of Software Engineering}
Software Engineering \textbf{is not just about coding}. An it integrates aspects of \underline{computer science}, \underline{project management} and \underline{engineering}.

\begin{itemize}
    \item Understands user needs and translates them into software solutions: \textbf{Requirements Engineering}.
    \item Designs systems that are robust, scalable and secure \textbf{(Quality Attributes)}.
    \item Managing the software development process efficiently.
\end{itemize}

\subsection{Software Development Lifecycle}
Software development lifecycle is a process used to develop software.

\begin{itemize}
    \item The \textbf{Waterfall Model} is a sequential software development process.
    \item \textbf{Phases:} Requirements, Design, Implementation, Verification, Deployment, Maintenance.
    \item Each phase cascades into the next like a waterfall, ensuring a structured and methodical approach.
\end{itemize} 
